\documentclass[11pt, a4paper]{article}
\usepackage[spanish]{babel}
\usepackage{amssymb}
\usepackage{enumerate}
\begin{document}
	\setcounter{section}{7}
	\section{Resumen}
	Esto concluye nuestro an\'alisis del diagrama de la secci\'on 2. Aqu\'i hay un resumen de los resultados clave que salieron del an\'alisis.(Ver Figura 1)
	\\\\
	\textbf{Teorema.} Sea $ABCD$ un cuadril\'atero c\'iclico con circuncentro $O$. Sea $P$ el punto intersecci\'on entre $AC$ y $BD$, las l\'ineas $AB$ y $CD$ se intersecan en $Q$, y las l\'ineas $DA$ y $CB$ se intersecan en $R$. La l\'inea $OP$ interseca a $QR$ en $M$. Luego
	\begin{enumerate}[(a)]
		\item Los circunc\'irculos de los siguientes tri\'angulos pasan por $M$: $QAD, QBC, RAB, RDC ,AOC, BOD.$ (En particular, $M$ es el Punto de Miquel del cuadril\'atero $ABCD$)
		\item $M$ es el centro de la semejanza espiral que traslada $A$ a $B$ y $B$ a $C$.
		\item $OM \perp QR$. De hecho, $M$ es el inverso de $P$ con respecto a el circunc\'irculo de $ABCD$
		\item El tri\'angulo $PQR$ es autopolar con respecto al circunc\'irculo de $ABCD$
		\\\\
		Recuerda esta configuraci\'on! Muchos problemas de geometr\'ia de olimpiadas son b\'asicamente una porci\'on de este gran diagrama.
	\end{enumerate}
\section{Problemas}
\begin{enumerate}
	\setcounter{enumi}{-1}
	\item Resuelve todos los ejercicios.
	\item (IMO, 1985) Un c\'irculo con centro $O$ pasa por los v\'ertices $A$ y $C$ de un tri\'angulo $ABC$ e interseca nuevamente los segmentos $AB$ y $BC$ en los puntos distintos $K$ y $N$, respectivamente. Los circunc\'irculos de los tri\'angulos $ABC$ y $KBN$ intersecan en exactamente dos puntos distintos $B$ y $M$.\\
	Probar que $\angle OMB = 90^{\circ}$
	
	\item (China 1992) El cuadril\'atero convexo $ABCD$ es inscrito en el c\'irculo $\omega$ con centro $O$. Las diagonales $AC$ y $BD$ se encuentran en el punto $P$. Los circunc\'irculos de los tri\'angulos $ABP$ y $CDP$ se intersecan en $P$ y $Q$. Asume que los puntos $O,P$ y $Q$ son distintos. Probar que $\angle OQP = 90^{\circ}$
	
	\item (Russia 1999) Un c\'irculo pasa por los v\'ertices $A$ y $B$ del tri\'angulo $ABC$ interseca al lado $BC$ nuevamente en el punto $D$. Un c\'irculo a trav\'es de $B$ y $C$ se encuentra con el lado el $AB$ en $E$ y con el primer c\'irculo otra vez en $F$. Probar que si los puntos $A, E, D, C$ est\'an en un c\'irculo con centro $O$, entonces $\angle BFO = 90^{\circ}$
	\item Los C\'irculos $\omega_1$ y $\omega_2$ se encuentran en los puntos $O$ y $M$. El c\'irculo $\omega$ con centro en $O$, se encuentra con los c\'irculos $\omega_1$ y $\omega_2$ en cuatro puntos distintos $A,B,C$ y $D$, tal que $ABCD$ es un cuadril\'atero convexo. Las l\'ineas $AB$ y $CD$ se encuentran en $N_1$. Las l\'ineas $AD$ y $BC$ se encuentran en el punto $N_2$. Probar que $N_1N_2 \perp MO$.
	
	\item (Russia 1995; TST Rumano 1996; Iran 1997) Considera un c\'irculo con di\'ametro $AB$ y centro $O$, Y sean $C$ Y $D$ dos puntos de este c\'irculo. La l\'inea $CD$ encuentra a la l\'inea $AB$ en el punto $M$ satisfaciendo $MB<MA$ y $MD< MC$. Sea $K$ el punto de intersecci\'on (Distindo a $O$) de los circunc\'irculos de los tri\'angulos $AOC$ y $DOB$. Muestre que $\angle MKO = 90^{\circ}$
	
	\item \begin{enumerate}[a.]
		\item Sean $A, B, C, D$ cuatro puntos distintos en el plano. Donde $AC$ y $BD$ se encuentran en el punto $P$, las l\'ineas $AB$ y $CD$ se encuentran en el punto $Q$, y las l\'ineas $BC$ y $DA$ se encuentran en $R$. La l\'inea que pasa por $P$ es paralela a $QR$ e interseca a $AB$ y $CD$ en $X$ y $Z$. Muestre que $P$ es el punto medio de $XZ$.
		\item Utiliza la parte (a) y el Hecho 8 para probar el Teorema de la Mariposa. Sea $C$ un c\'irculo y sea $EF$ una cuerda. $P$ es el punto medio de $EF$, y sean $AC$, $BD$ dos otras cuerdas que pasan por $P$. Suponer que $AB$ y $CD$ se encuentran con $EF$ en $X$ y $Z$, respectivamente, luego, $PX=PZ$
	\end{enumerate}
\item Sea $ABCD$ un cuadril\'atero c\'iclico con circuncentro $O$. Sean las l\'ineas $AB$ y $CD$ que se intersecan en $R$. Donde $\ell$ denota la l\'inea que pasa por $R$ perpendicular a $OR$. Probar que las l\'ineas $BD$ y $AC$ se encuentran en $\ell$ en puntos equidistantes de R.
\item (USA TST 2007) El tri\'angulo $ABC$ est\'a inscrito en el c\'irculo $\omega$. Las l\'ineas tangentes a $\omega$ en $B$ y $C$ se intersecan en el punto $T$. El punto $S$ se encuentra en el rayo $BC$ tal que $AS \perp AT$. Los puntos $B_1$ y $C_1$ se encuentran en el rayo $ST$ (Con $C_1$ entre $B_1$ y $S$ tal que $B_1T = BT = C_1T$. Probar que los tri\'angulos $ABC$ y $AB_1C_1$ son semejantes entre s\'i.
\item Sea $ABC$ un tri\'angulo con incentro $I$. Los puntos $M$ y $N$ son los puntos medios de los lados $AB$ y $AC$ respectivamente. Los puntos $D$ y $E$ se encuentran en las rectas $AB$ y $AC$, respectivamente, tal que $BD= CE = BC$. La recta $\ell_1$ pasa a trav\'es de $D$ y es perpendicular a la recta $IM$. La recta $\ell_2$ pasa a trav\'es de $E$ y es perpendicular a la recta $IN$. Sea $P$ la intersecci\'on de las rectas $\ell_1$ y $\ell_2$. Probar que $AP \perp BC$.
\item (IMO 2005) Sea $ABCD$ cuadril\'atero convexo con lados $BC$ y $AD$ de igual medida y no paralelos. Sean $E$ y $F$ puntos interiores de los lados $BC$ y $AD$ respectivamente, tal que $BE = DF$. Las rectas $AC$ y $BD$ se encuentran en el punto $P$, las rectas $BD$ y $EF$ se encuentran en $Q$, las rectas $EF$ y $AC$ se encuentran en $R$. Considera todos los tri\'angulos $PQR$ cuando $E$ y $F$ var\'ian. Muestre que los circunc\'irculos de estos tri\'angulos tienen un punto en com\'un distinto a $P$.
\item Un c\'irculo es inscrito en el cuadril\'atero $ABCD$ tal que toca los lados $AB, BC, CD, DA$ en $E,F,G,H$, respectivamente.
\begin{enumerate}[a.]
	\item Muestre que las rectas $AC, EF, GH$ son concurrentes. De hecho, concurren en el polo de $BD$.
	\item Muestre que las rectas $AC, BD, EG, FH$ son concurrentes.
\end{enumerate}
\item (China 1997) Sea el cuadril\'atero $ABCD$ inscrito en un c\'irculo. Suponga que las rectas $AB$ y $DC$ se intersecan en $P$ y las rectas $AD$ y $BC$ se intersecan en $Q$. Desde $Q$, construye dos tangentes $QE$ y $QF$ al c\'irculo d\'onde $E$ y $F$ son los puntos de tangencia. Probar que los tres puntos $P, E, F$ son colineales.
\item Sea un cuadril\'atero c\'iclico $ABCD$ con circuncentro $O$. Las rectas $AB$ y $CD$ se encuentran en $E$, $AD$ y $BC$ se encuentran en $F$, y $AC$ y $BD$ se encuentran en $P$. Luego, $EP$ y $AD$ se encuentran en $K$, y sea $M$ la proyecci\'on de $O$ hacia $AD$. Probar que $BCMK$ es c\'iclico.
\item (IMO Shortlist 2006) Los puntos $A_1, B_1$ y $C_1$ son elegidos en los lados $BC, CA$ y $AB$ de un tri\'angulo $ABC$, respectivamente. Los circunc\'irculos de los tri\'angulos $AB_1C_1, BC_1A_1,$ y $CA_1B_1$ intersecan al circunc\'irculo del tri\'angulo $ABC$ en los puntos $A_2, B_2, C_2$, respectivamente ($A_2 \ne A, B_2 \ne B, C_2 \ne C$). Los puntos $A_3, B_3$ y $C_3$ son sim\'etricos a $A_1, B_1, C_1$ con respecto a los puntos medios de los lados $BC, CA$ y $AB$, respectivamente. Probar que los tri\'angulos $A_2B_2C_2$ y $A_3B_3C_3$ son semejantes.
\item Punto de Euler de un cuadril\'atero c\'iclico
\setcounter{footnote}{3}
\begin{enumerate}[(a)]
	\item Sea un cuadril\'atero c\'iclico $ABCD$. Donde $H_A,H_B,H_C,H_D$ son los ortocentros de $BCD, ACD, ABD, ABC$, respectivamente. Muestre que $H_AH_BH_CH_D$ es la im\'agen de $ABCD$ bajo una reflexi\'on sobre alg\'un punto $E$ (i.e una rotaci\'on de $180^{\circ}$ sobre $E$).\\
	El Punto E es llamado el punto de Euler de $ABCD$ (Aparte: ¿Por qu\'e se llama el punto de Euler?\footnote{Pista: Recuerda que el punto de Euler de un tri\'angulo es otro nombre para el centro de la circunferencia de los nueve puntos})
	\item Muestre que $E$ se encuentra en la circunferencia de los nueve puntos de los tri\'angulos $ABC, ABD, ACD, BCD$.
	\item Muestre que $E$ se encuentra en la recta de Simson del tri\'angulo $ABC$ y el punto $D$.
	\item Muestre que $E$ tambi\'en es el punto de Euler de $H_AH_BH_CH_D$.
	\item Utilicemos $M_{XY}$ para denotar el punto medio de $XY$. Muestre que las perpendiculares de $M{AB}$ a $CD$, de $M_{BC}$ a $DA$, de $M_{CD}$ a $AB$, y de $M_{DA}$ a $BC$, concurren en $E$. 
\end{enumerate}
\end{enumerate}
\section{Pistas}
\begin{enumerate}
	\setcounter{enumi}{-1}
	\item Dale mi ciela, no es tan dif\'icil.
	\item Encuentra la configuraci\'on en el Gran Diagrama. El Hecho 8 es la clave.
	\item Esto es lo mismo que el problema anterior! (Por qu\'e?)
	\item ¡Ya lo hemos hecho demasiadas veces!
	\item Usa los Hechos 8 y 10.
	\item Mira el problema anterior. (Necesitamos que $AB$ sea el di\'ametro?)
	\item (a) Existe una soluci\'on de una sola l\'inea usando geometr\'ia proyectiva (Intenta utilizar una perspectiva en $Q$). (b) Utiliza $OP \perp QR$
	\item Teorema de la mariposa, metaforfizada.
	\item Para ver como esto entra en el Gran Diagrama, intenta usar $BCC_1B_1$ como el cuadril\'atero c\'iclico inicial.
	\item Repetidamente aplica el Hecho 9.
	\item Ves una semejanza espiral? D\'onde est\'a su centro?
	\item Utiliza el tri\'angulo autopolar diagonal de $EFGH$
	\item Utiliza el tri\'angulo autopolar diagonal de $ABCD$
	\item Con potencia de un punto, es suficiente mostrar que $FB \cdot FC = FM \cdot FK$
	\item Usa el Hecho 5, y observa que $\triangle C_2BA \sim \triangle C_1A_1B_1 \sim \triangle CA_3B_3$, paralelamente con los otros tres v\'ertices. Deduce que $\angle B_2A_2C_2 = \angle B_3A_3C_3$
	\item Los n\'umeros complejos puedes ser \'utiles. Para (c), recuerda lo siguiente: La recta de Simson de $ABC$ y $D$ biseca $DH_D$. Para (e), observa que la dilataci\'on de proporci\'on $2$ con centro en $B$ traslada a $M_AB$ a $A$ y a $E$ a $H_B$;.
\end{enumerate}
	
	
\end{document}